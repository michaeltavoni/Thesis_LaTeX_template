Lorem ipsum dolor sit amet, consectetur adipiscing elit. Integer ultrices ex a arcu mollis fermentum. Proin porttitor augue augue, et tempor eros eleifend id. Aliquam viverra at arcu et vulputate. Donec nec nibh sit amet ipsum imperdiet elementum ac non est. Cras et ante bibendum, auctor magna vitae, pretium neque. Vestibulum vel faucibus elit. Vestibulum eleifend dignissim nunc sed bibendum. Nunc scelerisque enim quis dolor lobortis, vitae porta orci pellentesque. Quisque nec leo et enim consequat consectetur nec et justo. Vestibulum ante ipsum primis in faucibus orci luctus et ultrices posuere cubilia curae; Quisque urna quam, sollicitudin eu laoreet id, faucibus vitae eros. In vitae nunc finibus, luctus ipsum blandit, ullamcorper nibh. Donec et porttitor nulla.\vspace{12pt}

\begin{table}[ht]
    \centering
    \begin{tabular}{|c|c|c|c|c|c|}
        \hline
        & \textbf{Var1} & \textbf{Var2} & \textbf{Var3} & \textbf{Var4} & \textbf{Var5} \\
        \hline
        \textbf{Var1} & 1.00 & 0.85 & 0.65 & 0.70 & 0.60 \\
        \hline
        \textbf{Var2} & 0.85 & 1.00 & 0.75 & 0.80 & 0.50 \\
        \hline
        \textbf{Var3} & 0.65 & 0.75 & 1.00 & 0.55 & 0.40 \\
        \hline
        \textbf{Var4} & 0.70 & 0.80 & 0.55 & 1.00 & 0.45 \\
        \hline
        \textbf{Var5} & 0.60 & 0.50 & 0.40 & 0.45 & 1.00 \\
        \hline
      \end{tabular}
      \caption{Tabella esempio}
      \label{tab:tabella-esempio}
\end{table}\vspace{12pt}

La Tabella \ref{tab:tabella-esempio} rappresenta la matrice di correlazione di un dataset fittizio.
\newpage %comando per interruzione di pagina

\begin{figure}[ht]
    \centering
    \includegraphics[width=0.8\textwidth]{immagine_esempio.png}
    \caption{Immagine di esempio}
    \label{fig:immagine-esempio}
\end{figure}

La Figura \ref{fig:immagine-esempio} mostra uno dei numerosi background di \textit{The Office}, una nota serie tv statunitense.\vspace{12pt}

Di seguito un elenco puntato:
\begin{itemize}
    \item Lorem ipsum
    \item Lorem ipsum
    \item Lorem ipsum
    \item ...
\end{itemize}

È possibile utilizzare anche un elenco numerato, come segue:
\begin{enumerate}
    \item Lorem ipsum
    \item Lorem ipsum
    \item Lorem ipsum
    \item Lorem ipsum
    \item ...
\end{enumerate}\vspace{12pt}

Come introdotto da Autore di Prova nel suo saggio di Prova \cite{articolo_prova}, questo è il modo per citare la bibliografia.